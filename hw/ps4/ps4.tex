\documentclass{article}

\usepackage{graphicx}
\usepackage{amsmath}
\usepackage[usenames]{color}
\usepackage{url}
\usepackage{courier}

\newcommand{\figref}[1]{Figure~\ref{#1}}
\newcommand{\widgraph}[2]{\includegraphics[keepaspectratio,width=#1]{#2}}
\pagestyle{empty} \addtolength{\textwidth}{1.0in}
\addtolength{\textheight}{0.5in} \addtolength{\oddsidemargin}{-0.5in}
\addtolength{\evensidemargin}{-0.5in}
\newcommand{\ruleskip}{\bigskip\hrule\bigskip}
\newcommand{\nodify}[1]{{\sc #1}} \newcommand{\points}[1]{{\textbf{[#1
points]}}}
\newcommand{\bitem}{\begin{list}{$\bullet$}%
{\setlength{\itemsep}{0pt}\setlength{\topsep}{0pt}%
\setlength{\rightmargin}{0pt}}} \newcommand{\eitem}{\end{list}}
\newcommand{\comment}[1]{}
\newcommand{\true}{\mbox{true}}
\newcommand{\Parents}{\mbox{Parents}}
\newcommand{\eat}[1]{}
\newcommand{\CInd}[3]{({#1} \perp {#2} \mid {#3})}
\newcommand{\Ind}[2]{({#1} \perp {#2})}
\setlength{\parindent}{0pt} \setlength{\parskip}{0.5ex}

%%%%%%%%%%%%%%%%%%%%%%%%%%%%%%%%%%%%%%%%%%%%%%%%%%%%%%%%%%%%%%%%%%%%

\begin{document}

\pagestyle{myheadings} \markboth{}{DS-GA-1005/CSCI-GA.2569 Problem Set 4}

{\LARGE 
\begin{center}Inference and Representation, Fall 2016\end{center}
}

{\Large
Problem Set 4: Gibbs sampling
}

{\bf Due: Monday, October 17, 2016 at 3pm} (uploaded to Gradescope/NYU Classes.)\\

{\bf Your submission should include a PDF file called ``solutions.pdf''
  with your written solutions, separate output files, and all of the
  code that you wrote.}

{\em {\bf Important:} See problem set policy on the course web site.}
\ruleskip


\newcommand{\eparam}{\ensuremath{\eta}}
\newcommand{\meanpar}{\ensuremath{\mu}}
\newcommand{\Exs}{\ensuremath{\mathbb{E}}}
\newcommand{\bksl}{\ensuremath{\backslash}}
\newcommand{\mprob}{\ensuremath{\mathbb{P}}}
\newcommand{\indmark}{\ensuremath{\perp}}
\newcommand{\graph}{\ensuremath{G}}
\newcommand{\vertex}{\ensuremath{V}}
\newcommand{\estim}[1]{\ensuremath{\widehat{#1}}}
\newcommand{\myxsam}[1]{\ensuremath{x_{#1}}}
\newcommand{\myysam}[1]{\ensuremath{y_{#1}}}
\newcommand{\ypred}[1]{\ensuremath{\widetilde{y}^{(#1)}}}

\def\b{\backslash}

\def\child{\xi}

%%%%%%%%%%%%%%%%%%%%%%%%%%%%%%%%%%%%%%%%%%%%%%%%%%%%%%%%%%%%%%%%%%%%%

%%%%%%%%%%%%%%%%%%%%%%%%%%%%%%%%%%%%%%%%%%%%%%%%%%%%%%%%%%%%%%%%%%%%

\begin{enumerate}
\item {\bf Conjugacy and Bayesian prediction} (generalization of
  Bernoulli example from Murphy 9.2.5.5):
\begin{enumerate}
\item Let $\theta \sim \mbox{Dir}(\alpha)$. Consider discrete random variables 
$(X_1, X_2, \ldots, X_N)$, where $X_i \sim \mbox{Cat}(\theta)$ for each $i$ 
(thus the $X_i$ are conditionally independent of one another given
$\theta$). 
Show that the posterior $\Pr(\theta \mid x_1, \ldots, x_N, \alpha)$ is given by $\mbox{Dir}(\alpha')$, where
\[
\alpha'_k = \alpha_k + \sum_{i=1}^N 1[x_i = k].
\]
This property, that the posterior distribution $\Pr(\theta \mid {\bf
  x})$ is in the same family as the prior distribution $\Pr(\theta)$,
is called {\em conjugacy}. The Dirichlet distribution (see
Murphy Sec. 2.5.4) is the {\em conjugate prior} for the Categorical distribution. Every distribution in the exponential family has a conjugate prior. For example, the conjugate prior for the mean of a Gaussian distribution can be shown to be another Gaussian distribution.
% TODO: If the homework were longer, I would introduce the general form of a conjugate prior for a distribution in the exponential family and have the students derive the Dirichlet as the conjugate prior for the multinomial.

\item Now consider a random variable $X_{\mbox{\rm new}} \sim \mbox{Cat}(\theta)$ 
that is assumed conditionally independent of $(X_1, X_2, \ldots, X_N)$ 
given $\theta$.  Compute:
\[
p(x_{\mbox{\rm new}} \mid x_1, x_2, \ldots, x_N, \alpha)
\]
by integrating over $\theta$.\\

{\em Hint}: Your result should take the form of a ratio of gamma functions.\\

This is called {\em Bayesian} prediction because we put a prior distribution over the parameters $\theta$ (in this case, a Dirichlet) and are thus able to take into consideration our initial uncertainty over (and prior knowledge of) the parameters together with the evidence we observed (samples $x_1, \ldots, x_N$) when giving our predictions for $x_{\mbox{\rm new}}$.
\end{enumerate}

\item 
Latent Dirichlet allocation (LDA) is a probabilistic model for discovering topics in
sets of documents \cite{blei2003latent}.  The generative model is as follows:
\begin{itemize}
  \itemsep 5pt
  \item For each document, $m = 1, \ldots, M$
    \begin{enumerate}
      \item Draw topic probabilities $\theta_m \sim p(\theta | \alpha)$
      \item For each of the $N$ words:
        \begin{enumerate}
        \item Draw a topic $z_{mn} \sim p(z | \theta_m)$
        \item Draw a word $w_{mn} \sim p(w | z_{mn}, \beta)$,
        \end{enumerate}
    \end{enumerate}
\end{itemize}
where $p(\theta | \alpha)$ is a Dirichlet distribution, and where $p(z |
\theta_m)$ and $p(w | z_{mn}, \beta)$ are Multinomial distributions.  Treat
$\alpha$ and $\beta$ as fixed hyperparameters.
Note that $\beta$ is a matrix, with one column per topic, and the
Multinomial variable $z_{mn}$ selects one of the columns of $\beta$ to yield
multinomial probabilities for $w_{mn}$. 


\begin{figure}[t]
\begin{center}
  \widgraph{.5\textwidth}{lda}
\caption{Graphical structure of the LDA model.}
\end{center}
\end{figure}

\begin{enumerate}

\item In this question you will use an off-the-shelf implementation of
  LDA to get practice with learning topic models on real-world data,
  and to analyze various trade-offs that can be made during
  learning. 
\begin{enumerate}
\item Prepare a corpus of documents from which you'll learn. You can
  find some already prepared text collections here:

\begin{centering}\url{https://archive.ics.uci.edu/ml/datasets/Bag+of+Words}\end{centering}

However, we prefer that you be creative and construct your own!

\item Learn a latent Dirichlet allocation model on your corpus using
  default parameters. You can use any software package that you like. Two excellent
  options are:
\begin{itemize}
\item Mallet (\url{http://mallet.cs.umass.edu/})
\item Gensim (\url{http://radimrehurek.com/gensim/})
\end{itemize}
Qualitatively describe what topics are discovered.

\item Re-run learning using varying numbers of topics (e.g., 5, 20,
  100). Describe qualitatively the differences that you observe 
  as the number of topics increases.
\end{enumerate}

\item Derive a Gibbs sampler for the LDA model (i.e., write down the
set of conditional probabilities for the sampler; see Sec. 24.2 of
Murphy). To obtain full credit, you must hand in your full derivation,
not just the final formulas.

{\em You may find it helpful to refer to your solutions from question 1.}

\item Derive a collapsed Gibbs sampler for the LDA model,
  where you consider the marginal distribution $\Pr({\bf z}_m \mid
  {\bf w}_m ; \alpha, \beta)$ (integrating out {\em just} the topic probabilities
  $\theta_m$; here we assume that $\beta$ is known) and are now only sampling ${\bf z}$. Again, you must
  hand in your full derivation.

\item 
Implement both of the inference algorithms that you derived. You will then run your algorithms to find the posterior topic distribution $\theta$ for an input document.

We have previously learned the parameters (i.e., $\alpha$ and $\beta$) of a 200-topic LDA model on a corpus containing thousands of abstracts of papers from the top machine learning conference, Neural Information Processing Systems (NIPS). Your task will be to infer the topic distribution for a new document.

We have provided the following data files:
\begin{itemize}
      \item \texttt{alphas.txt}, which has on each line for topic $i$: $i$, $\alpha_i$, and a list of the most likely words for this topic,
      \item \texttt{abstract\_*.txt}, with the words of document $m$ (i.e., the abstract),
      \item \texttt{abstract\_*.txt.ready}, with, in order,
      \begin{itemize}
           \item the number of topics $k$,
           \item $\alpha_i$, for $i = 1, \ldots, k$,
           \item for every word $w_n$, the word itself followed by $\beta_{w_n,i}$ 
           for $i = 1, \ldots, k$.
      \end{itemize}
\end{itemize}
Note that your code only needs to read in the \texttt{abstract\_*.txt.ready} files -- the \texttt{alphas.txt} and \texttt{abstract\_*.txt} files are provided for your reference only.

It is common with MCMC methods to discard the first $X$ samples to avoid using samples that are highly correlated with the arbitrary starting assignment (this is called ``burning in''). Use $X=50$ for your Gibbs sampling implementations.

For each of the abstracts,
\begin{enumerate}
\item Use your code to generate an accurate estimate of $E[\theta]$ using collapsed Gibbs sampling with a high number of iterations (e.g. $10^4$). Use this as ground truth. % for answering the subsequent questions.

The following formula can be used to obtain an estimate of $\theta$ from the collapsed Gibbs sampler (where $T$ is the number of samples):
\[
E[\theta_i] = \frac{T\alpha_i + \sum_{t=1}^T\sum_{n=1}^N 1[z^t_{n} = i]}{T(\sum_{\hat{i}=1}^k \alpha_{\hat{i}} + N)}
\]

\item Plot the $\ell_2$ error on your estimate of $E[\theta]$ as a function of the number of iterations for each of the algorithms.

\end{enumerate}

{\bf Only include in your solutions the plot for the data file} \texttt{NIPS2008\_0517}. The remaining files are provided for your own experimentation.

You may use the programming language of your choice. We recommend first checking that packages are available to (1) sample from a Dirichlet distribution, and (2) compute the Digamma function $\Psi(x)$, as these will simplify your coding. For example, see Python's \texttt{numpy.random.mtrand.dirichlet} and \texttt{scipy.special.psi}.

\end{enumerate}
\end{enumerate}

\bibliographystyle{plain}
\bibliography{../refs}

%%%%%%%%%%%%%%%%%%%%%%%%%%%%%%%%%%%%%%%%%%%%%%%%%%%%%%%%%%%%%%%%%%%%%%%%%%

\end{document}

